%%%%%%%%%%%%%%%%%%%%%%%%%%%%%%%%%%%%%%%%%%%%%%%%%%%%%%%%%%%%%%%%%%%%%%%%%%%%%%%%


%\documentclass[letterpaper, 10pt, conference]{IEEEtran}
\documentclass{beamer}
\usepackage{tikz}
\usetikzlibrary{calc,decorations.markings,arrows, fadings}
\usepackage{tikz-timing}
\usepackage{amsmath}
\usepackage{xifthen}
\usepackage{amsfonts}
\usepackage{tabu}
\usepackage{graphicx,dblfloatfix}
\usepackage[font=small]{caption}
%\usepackage{subcaption}
\usepackage{cite}
\usepackage{placeins}
\usepackage{xspace}
\usepackage{subfiles}
\usepackage{url}
\usepackage[outline]{contour}

\contourlength{0.18em}

\usetheme{Copenhagen}
\usecolortheme{beetle} 
\setbeamertemplate{navigation symbols}{}
\setbeamerfont{footline}{series=\scriptsize}
\usefonttheme{structurebold}

\definecolor{beetle@other}{RGB}{207,184,124} %cu gold
%
%Once I break this in to multiple sections, each section will be added with:
%	\subfile{filename}
%In those other files, I have as preamble:
%	\documentclass[main.tex]{subfiles}
%

\title[\textcolor{black}{Low-Cost Omnidirectional Powertrain}]{A Stick-Slip Omnidirectional Powertrain for Low-Cost Swarm Robotics:\\ Mechanism, Calibration, and Control}
\author[{\tt john.klingner@colorado.edu}]{John Klingner, Anshul Kanakia, Nicholas Farrow, Dustin Reishus and Nikolaus Correll}
\institute[]{Department of Computer Science\\University of Colorado Boulder}
\date{}

\newcommand{\Tau}{\boldsymbol{\mathrm{T}}}

\begin{document}
\begin{frame}
	\titlepage
\end{frame}
\begin{frame}
	\frametitle{Motivation}
	\begin{itemize}
		\item As part of the general exploration of ways to get low-cost robots moving, some work has been done over the past decade on ``stick-slip'' motion.
		\item This motion principle uses the impulse forces generated by low-cost vibration motors.
		\item Previous work has demonstrated the efficacy of this approach for skid steering motion driven by two motors.**CITE** This work determined that a three-motor setup was infeasible in real robots.**CITE**
		\item With a small change in methods, we demonstrate the feasibility of an omnidirectional powertrain using three motors.
	\end{itemize}
\end{frame}
\begin{frame}
	\frametitle{The Method}
	\begin{itemize}
		\item Previous work mounts motors near the legs of the robot. Our method requires that the motors be mounted opposite the legs. **FIGURE** **TALK TO NICK/DUSTIN ABOUT WHAT HAPPENS WHEN THEY AREN'T**.
		\item The problem observed by previous work was that using three motors at once set up vibrations throughout the robot resulting in instability and uncontrolled motion.	
		\item To avoid this issue, only one motor is ever activated at any given time. **THOUGH I WONDER WHAT HAPPENS IF WE TURN ON TWO AT ONCE?**
		\item One rotation of one motor causes the robot to pivot about the opposite leg, we refer to this as a `step'.
		\item Overall motion is achieved with a sequence of these steps.
	\end{itemize}
\end{frame}
\end{document}