%%%%%%%%%%%%%%%%%%%%%%%%%%%%%%%%%%%%%%%%%%%%%%%%%%%%%%%%%%%%%%%%%%%%%%%%%%%%%%%%


%\documentclass[letterpaper, 10pt, conference]{IEEEtran}
\documentclass[letterpaper, 10pt, conference]{ieeeconf}
\IEEEoverridecommandlockouts
\overrideIEEEmargins 
\usepackage{tikz}
\usetikzlibrary{calc,decorations.markings,arrows}
\usepackage[top=1in,bottom=1in,right=1in,left=1in]{geometry}
\usepackage{amsmath}
\usepackage{xifthen}
\usepackage{amsfonts}
\usepackage{tabu}
\usepackage{graphicx}
\usepackage[font=small]{caption}
\usepackage{subcaption}
\usepackage{cite}
\usepackage{xspace}
\usepackage{subfiles}



% Nice Little macro for adding a comment box. Include incrementing comment numbers.
\newcounter{comcount}
\setcounter{comcount}{0}
\newcommand{\mycomment}[1]
{
\refstepcounter{comcount}
\smallskip\noindent\fbox{\parbox{\linewidth}{\emph{Comment \arabic{comcount}} : \small{#1}}} 
}

%
%Once I break this in to multiple sections, each section will be added with:
%	\subfile{filename}
%In those other files, I have as preamble:
%	\documentclass[main.tex]{subfiles}
%

\title{\LARGE \bf
A Stick-Slip Omnidirectional Drive-train for Low-Cost Swarm Robotics
}

\author{John Klingner and ??? and Nicholas Farrow, Dustin Reishus and Nikolaus Correll%
\thanks{Department of Computer Science,
University of Colorado at Boulder,
 Boulder, CO 80309,
{\tt\small firstname.lastname{@}colorado.edu}}%
}

\begin{document}
\maketitle


%%%%%%%%%%%%%%%%%%%%%%%%%%%%%%%%%%%%%%%%%%%%%%%%%%%%%%%%%%%%%%%%%%%%%%%%%%%%%%%%
\begin{abstract}
ABSTRACT GOES HERE
\end{abstract}



%%%%%%%%%%%%%%%%%%%%%%%%%%%%%%%%%%%%%%%%%%%%%%%%%%%%%%%%%%%%%%%%%%%%%%%%%%%%%%%%
\section{Introduction}
We present a low-cost miniature robot drive-train that approximates omni-directional motion. Classically, miniature robotic platforms such as r-one \cite{mclurkin2013low}, Jasmine \cite{jasmine} and Alice \cite{alice} require geared motors, which are expensive and difficult to miniaturize. The proposed drive-train is based on the ``stick-slip'' actuator that has been introduced in \cite{breguet1998stick}, and has been shown to be particularly attractive for high precision movements \cite{brufau2005micron,chu2006novel,martel2001three,martel2005fundamental,eigoli2012locomotion} and force control \cite{vartholomeos2008analysis}.   

Due to its simplicity --- only vibration motors are required --- the stick-slip principle is also increasingly been used on low-cost miniature robot platforms such as the Kilobots \cite{rubenstein2012kilobot} based on the design presented in \cite{Vartholomeos2006}, which approximates the dynamics of a differential wheel platform (see also \cite{spartali2013speed} for additional analysis). Achieving fully holonomic motion on the plane requires at least three vibration motors, however. Whereas \cite{Vartholomeos2005} proposes a symmetric three motor design and its analysis, their approach is not experimentally feasible. This paper presents, for the first time an experimentally verified design and shows how challenges such as variability between motors caused by manufacturing can be compensated using an optimization-based online calibration. 

\begin{figure}[h]
	\centering
		\includegraphics[width=0.8\columnwidth]{./Images/droplets.png}
	\caption{The Droplet swarm robotics platform. Two Droplets are shown, one with the cover removed. A penny is included for scale. The background is the floor of alternating power and ground strips from which the Droplets draw power.}
	\label{droplets}
\end{figure}

\subsection{Related work}
Other miniature robotic platforms such as e-puck, Jasmine and Alice \cite{}. They require geared motors, which are expensive and difficult to miniaturize. Other stick-slip robots, the nanobots from MIT, the kilobots \cite{}. They are cheap, easy to manufacture, but not fully holonomic. In this work, we present a 3-leg stick-slip locomotion system that is fully holonomic. 



%%%%%%%%%%%%%%%%%%%%%%%%%%%%%%%%%%%%%%%%%%%%%%%%%%%%%%%%%%%%%%%%%%%%%%%%%%%%%%%%
\section{The Droplet Swarm Robotic Platform}
The Droplets are an open-source swarm robotic platform, with source code and manufacturing information available online. Their body has a radius of $2.2cm$. They have three extended headers for legs located symmetrically around the robot $1.5cm$ from the center. The Droplets receive power from these legs through a floor with alternating strips of $+5V$ and $GND$. The Droplets use an Atmel Xmega128A3U microcontroller, along with Allegro A3901 Dual Motor Drivers. The motors are a low cost, coin-type model and are mounted symmetrically around the robot opposite the legs, near the outer edge. The motors are oriented such that the axis of rotation moves through the center of the robot and the opposite leg. Due probably to their low cost, our motors are inconsistent about the direction in which they spin, something that must be addressed by our auto-calibration process.

\begin{figure}[h]
	\centering
	\subfile{motorLocations}
	\caption{An image of the Droplet's shell, with just the motors included. The locations where the legs are mounted are labeled.}
	\label{motorLocations}
\end{figure}



%%%%%%%%%%%%%%%%%%%%%%%%%%%%%%%%%%%%%%%%%%%%%%%%%%%%%%%%%%%%%%%%%%%%%%%%%%%%%%%%
\section{Principle of Operation}

\subsection{Base Motion Principle}
A vibration motor is simply a motor with a mass on the shaft, such that the center of mass is not on the axis of rotation. Spinning the motor ``throws the mass around'', causing motion of the motor. To understand why this works, note that as the mass swings through its circular path, the motor experiences a force towards that mass. Over the course of a full rotation, the net force experienced by the motor is 0. See Figure~\ref{motorDiagram} helps to illustrate why this works.

\begin{figure}
\centering
\subfile{singleDoFModel}
\caption{Simple 1DoF model of motion principle, from the side.}
\label{motorDiagram}
\end{figure}

To achieve 1DoF motion using this principle, we must mount the motor on some sort of platform. If we neglect friction, the platform will slide backwards and forwards due to the backwards and forwards forces from the mass, but experiences no net translation. With friction, the vertical forces of the mass become relevant. The downward force of gravity is mitigated by the upward force from the swinging mass. Crucially, this means that the total downward force experienced by the platform is lower while the mass is swinging upwards. Thus, the platform experiences reduced friction while the mass is swinging upwards. The direction of lateral magnitude during this period of reduced friction (or, the direction the motor is spinning) determines the direction of travel. \cite{Vartholomeos2005,Vartholomeos2006} go in to more mathematical detail of this model.




\subsection{Motion Principle Applied to Droplets}

Now we will take that 1DoF model and apply it to our 3DoF robotics platform. See Figure~\ref{dropletMotorDiagram} for reference. Here, we deviate from \cite{Vartholomeos2005} by positioning our motors opposite the platform's legs. With the simplifying assumption that the platform is a uniform mass, lets consider the forces applied to the platform as a single motor rotates. Since the platform touches the floor only with its three legs, it will be relevant to consider the downward force on each leg separately ($f_i$ for forces, $m_i$ for motors?)

\begin{figure}
\centering
\subfile{dropletDiagram}
\caption{Droplets motion principle, now looking from above.}
\label{dropletMotorDiagram}
\end{figure}

As the mass on the motor swings upwards, the friction experienced by each leg is reduced. Moreover, it is reduced more for the two legs nearest the motor. The lateral force by the motor -- which is tangent to the platform's circular frame -- thus causes the platform to pivot about the leg opposite it.



\subsubsection{Walking Straight}

Let us consider a single pivot caused by motor 1 ($m_1$), about leg 1 ($l_1$), of some small, positive radial distance $\theta$. This motion causes the center of the Droplet to trace $\theta$ of an arc about $l_1$. The length of the chord with the same endpoints as the arc, $C$, ie. the distance traveled by the center of the Droplet, is given by:
\[
C=2 L \sin\left(\frac{\theta}{2}\right)
\]
Where $L$ is the distance between the robot's center and leg; the radius of the arc. To break this $c$ displacement in to its $x$ and $y$ components, we first need to know the angles of the triangle made by the chord and the leg. We can get this using the law of sines:
\[
x = \arcsin\left(\frac{L}{C}\sin(\theta)\right) 
\]
Inserting our equation for $C$ gives:
\begin{align*}
x &=\arcsin\left(\frac{L\sin(\theta)}{2 L \sin(\frac{\theta}{2})}\right) \\
   &=\arcsin\left(\cos\left(\frac{\theta}{2}\right)\right)
\end{align*}

Given this angle $x$, and the known angle between the Droplet's $x$\~axis and the leg ($\phi_1$), we can calculate the direction of the platform's displacement:

\begin{align*}
\delta x_a &= C \cos(\pi - x - \phi)\\
\delta y_a &= C \sin(\pi - x - \phi)
\end{align*}

Note also that the orientation of the Droplet has changed by $\theta$. Next, consider a second pivot by $m_2$ of $-\theta$. By symmetry, we now have:

\begin{align*}
\delta x_b &= -C\cos(\pi -x -\phi)\\
\delta y_b &= C \sin(\pi - x - \phi)
\end{align*}

and the orientation has now changed by $-\theta$, putting the Droplet back in its original orientation. 

***ROUGH***
Having trouble with the formalism here, but the basic idea is that, the Droplet IS NOT going straight - it's going at an angle, because the orientation it takes the second step at is different than the orientation of the first step. The size of this error is a function of $\theta$. As $\theta$ approaches $0$, so does the error. Similarly, when spinning each cycle of three motors doesn't leave us back at our exact original position, but it does bring us pretty close.

By restricting ourselves to these small-pivot ``steps'', we avoid the requirements of precise balance and motor-phase syncronization that causes trouble with more direct implementations \cite{Vartholomeos2006}.

To simplify, we focus on the platform moving in one of six directions: (0,180, $+/-$60, $+/-$120) degrees, as well as clockwise and counterclockwise rotation. It should be a straightforward extension to get arbitrary 3DoF motion by combining these directions.

There are still some difficulties to overcome with physical implementation. In particular, requiring that each step be of consistent size causes trouble. Variations in the manufacturing of our motors and platform mean that the robot must be calibrated to ensure consistent step size.




\subsection{Forward and inverse kinematics}
With the control parameters $\dot{theta}_1$, $\dot{theta}_2$ and $\dot{theta}_3$ that are due to vibration of motors $m_1$, $m_2$, and $m_3$, respectively, we can derive the following forward kinematic equations in the robot coordinate frame $\dot{x}_r$, $\dot{y}_r$ and $\dot{\theta}_r$:
Here we assume the y-axis to be orthogonal to the spinning direction of motor 2, and the y-axis to to split the angle between the axis connecting motor 1 and leg 1, and the axis connecting motor 0 and leg 0.
\begin{eqnarray}\label{eq:fwd}
\dot{x}_r &=& L\dot{\theta_2}-L\dot{\theta_1}\frac{\sqrt{3}}{2}+L\dot{\theta_0}\frac{\sqrt{3}}{2}\\
\dot{y}_r &=& L\dot{\theta_0}\frac{\sqrt{3}}{2} - L\dot{\theta_1}\frac{\sqrt{3}}{2}\\
\dot{\theta}_r &=& \dot{\theta_0} + \dot{\theta_1} + \dot{\theta_2}
\end{eqnarray}
Here, we assume positive $\theta_i$ to be aligned with the direction of thrust of motor $m_i$ and each leg having distance $l$ from the motors. That is, if all motors spin in the same direction, they all contribute to $\dot{\theta}_r$ in the same way. The fraction $\frac{\sqrt{3}}{2}$ is equivalent from $\cos 30^o$. Notice that motor 2 does not contribute to $\dot{y}_r$ as it spins orthogonally to the y-axis.

Equation \ref{eq:fwd} is a linear system with three unknowns, and we can solve
\begin{eqnarray}
\dot{\theta_2}=\frac{(\dot{x}-\dot{y})}{L}
\end{eqnarray}
and so on.



%%%%%%%%%%%%%%%%%%%%%%%%%%%%%%%%%%%%%%%%%%%%%%%%%%%%%%%%%%%%%%%%%%%%%%%%%%%%%%%%
\section{Experimental platform}
We constructed the ``Droplets'' platform, a ping-pong ball sized, spherical robot with a flat bottom, three legs, three coin-type vibration motors arranged as shown in Figure \ref{dropletMotorDiagram}, and a 6-directional infrared range and bearing system \cite{farrow14}. The legs are spaced 120 degrees apart 15.1mm (0.595in) from the center.

The vibration motors (Hochar) are circular shaped, 10mm in diameter and 2.7mm thick, weigh 1.1g, rated at 3V DC, vibrate at 200$\pm$42 Hz, and of the type commonly used in cell phones and pagers. Although equipped with color-coded wires, the direction of spin is arbitrary and differences in vibration frequency and strength are significant. 

Motors are controlled via an H-Bridge (Allegro A3901) that is driven by a pulse-width modulated (PWM) signal generated by a microcontroller (Atmel Xmega123A3). This setup allows us to provide a series of precise pulses as well as changing the direction of rotation.    

***TODO: I think it is important to explain all the parameters given to optimization algorithm and the values the it returns + how they are used to optimize motor values (figures of PWM and duty cycles and resolution, etc, etc.).***


%%%%%%%%%%%%%%%%%%%%%%%%%%%%%%%%%%%%%%%%%%%%%%%%%%%%%%%%%%%%%%%%%%%%%%%%%%%%%%%%
\section{Auto-Calibration}

It is possible to calibrate each platform manually, by setting values and eyeballing how well they work. As stated earlier, however, this platform is motivated by a desire for low-cost, large scale robot swarms.  Manual calibration of a swarm of thousands of robots would be completely impractical, and so we automate the process.

\subsection{Experimental Setup}
Data is collected using a high resolution USB camera\footnote{Microsoft LifeCam Cinema}, and RoboRealm \cite{RoboRealm}. A fiducial marking is taped to the top of the platform to aid in tracking and orientation calculations. Specifically, we collect the position (x,y coordinate) and orientation of the Droplet in each frame.

To update the Droplet's calibration settings, we find the least-squares best-fit circle of the collected coordinates. Specifically, the circle's radius and whether it is on the Droplet's left or right. While calibrating for walking straight, we optimize for maximizing the radius of the fitted circle (if the Droplet travelled perfectly straight, the radius would by infinite). While calibrating for spinning, we optimize for minimizing the radius of the fitted circle (a perfect spin would have radius 0).

We communicate with and control the platform through a serial to IR transmitter. 

***IMAGE OF EXPERIMENT SETUP WOULD BE NICE***


\subsection{Calibration Method}
Provide method overview.

Use the popular Nelder-Mead Simplex algorithm \cite{NelderMead} to calibrate motor parameters for spinning in a given direction (clockwise or anti-clockwise). Each vertex in the simplex is a set of three motor calibration parameters. The simplex algorithm attempts to minimize the radius of the circle that the droplet makes while rotating while, at the same time, maximize the speed of rotation. The objective function is therefore,
\begin{equation}
\text{(min.)} \quad -r_{circ}(\mu_1, \mu_2, \mu_3) + \alpha.v_{rot}(\mu_1, \mu_2, \mu_3)
\end{equation}
where $\alpha$ is a weighting constant set to a value less than 1 as we prefer motor values that minimizing circle radius over rotational speed.

The objective function given to Nelder-Mead simplex is, minimize the radius of the circle fitted to the points on the Droplet's path as it spins.

There are many sets of parameters that will all give a small radius - as long as all pivots are of the same magnitude. This just gives one of them.

Additionally, this method identifies which motors are `flipped'---i.e., which motors spin opposite from what's expected. This is because having the motor directions wrong results in a large radius.

It's possible that Nelder-Mead will find a CCW spin, with all motors flipped. After the algorithm comes to a stop, the mean change in orientation is checked to verify that the spin direction is correct. The rest of the procedure only requires that one direction of spin be optimized; it does not care which.

Once we have clockwise spin calibrated, this is equivalent to having three matched pivot sizes for the motors' clockwise spins. To finish calibrating, we just need to find counterclockwise values that match the clockwise ones.

This is done for each motor independently by having the Droplet walk straight in a direction. Now, we want to maximize the radius of the the circle traced by the Droplet. As the droplet improves straight-line motion accuracy, the radius of the circle transcribed by its motion approaches $\infty$. Changes in calibration values are made proportional to circle radius until we've reached the limits imposed by the resolution of our calibration value.

Simply calibrating for CW and then CCW spin doesn't work because it gives two sets of pivot magnitudes, one for CW and one for CCW rotation. Each set is matched with the other pivots in its set, but without additional information we have no information about the comparative magnitudes across the sets.




%%%%%%%%%%%%%%%%%%%%%%%%%%%%%%%%%%%%%%%%%%%%%%%%%%%%%%%%%%%%%%%%%%%%%%%%%%%%%%%%
\section{Results}

What are the experiments that we have done to show that this works? All plots here.

Use CCW spin to verify results.




%%%%%%%%%%%%%%%%%%%%%%%%%%%%%%%%%%%%%%%%%%%%%%%%%%%%%%%%%%%%%%%%%%%%%%%%%%%%%%%%
\section{Discussion}
Ditch the whole computer vision set up and use the Droplet's IR communication and range and bearing powers to get data and have the swarm calibrate itself.

Get arbitrary 3DoF motion by combining directions.

\section{Conclusion}
To make an analogy to animal strides, the Droplets are only capable of strides in which one leg is ``in the air'' at a time. Any more than that and the Droplet becomes unstable.

Conclusory stuff here.

\section*{Acknowledgments}
This work has been supported by NSF award \#1150223 and a CI fellowship to D. Reishus.


\bibliographystyle{ieeetr}
\bibliography{mybibfile}

\end{document}

